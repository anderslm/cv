%%%%%%%%%%%%%%%%%
% This is an sample CV template created using altacv.cls
% (v1.1.4, 27 July 2018) written by LianTze Lim (liantze@gmail.com). Now compiles with pdfLaTeX, XeLaTeX and LuaLaTeX.
% 
%% It may be distributed and/or modified under the
%% conditions of the LaTeX Project Public License, either version 1.3
%% of this license or (at your option) any later version.
%% The latest version of this license is in
%%    http://www.latex-project.org/lppl.txt
%% and version 1.3 or later is part of all distributions of LaTeX
%% version 2003/12/01 or later.
%%%%%%%%%%%%%%%%
\newcommand{\RNum}[1]{\uppercase\expandafter{\romannumeral #1\relax}}
%% If you need to pass whatever options to xcolor
\PassOptionsToPackage{dvipsnames}{xcolor}

%% If you are using \orcid or academicons
%% icons, make sure you have the academicons 
%% option here, and compile with XeLaTeX
%% or LuaLaTeX.
% \documentclass[10pt,a4paper,academicons]{altacv}

%% Use the "normalphoto" option if you want a normal photo instead of cropped to a circle
% \documentclass[10pt,a4paper,normalphoto]{altacv}

\documentclass[10pt,a4paper]{altacv}
%% AltaCV uses the fontawesome and academicon fonts
%% and packages. 
%% See texdoc.net/pkg/fontawecome and http://texdoc.net/pkg/academicons for full list of symbols.
%% 
%% Compile with LuaLaTeX for best results. If you
%% want to use XeLaTeX, you may need to install
%% Academicons.ttf in your operating system's font 
%% folder.

% Change the page layout if you need to
\geometry{left=1cm,right=9cm,marginparwidth=6.8cm,marginparsep=1.2cm,top=1.25cm,bottom=1.25cm,footskip=2\baselineskip}

% Change the font if you want to.

% If using pdflatex:
\usepackage[T1]{fontenc}
\usepackage[utf8]{inputenc}
\usepackage[default]{lato}

% If using xelatex or lualatex:
% \setmainfont{Lato}

% Change the colours if you want to
\definecolor{Navy}{HTML}{000080}
\definecolor{SlateGrey}{HTML}{2E2E2E}
\definecolor{LightGrey}{HTML}{444444}
\colorlet{heading}{Navy}
\colorlet{accent}{Navy}
\colorlet{emphasis}{SlateGrey}
\colorlet{body}{LightGrey}

\usepackage[colorlinks]{hyperref}

\begin{document}

\name{Anders Ladegaard Marchsteiner}

\tagline{Systems developer}
\personalinfo{%
  \birthday{19/11 1986}
  \email{alm.anma@gmail.com}
  \phone{(+45) 4292 7284}
  \location{Bog\o\ by, Denmark}
  \linkedin{anders-marchsteiner}
  \github{anderslm} 
}


%% Make the header extend all the way to the right, if you want. 

\begin{fullwidth}
\makecvheader
\end{fullwidth}

%% Depending on your tastes, you may want to make fonts of itemize environments slightly smaller
% \AtBeginEnvironment{itemize}{\small}


%% Provide the file name containing the sidebar contents as an optional parameter to \cvsection.
%% You can always just use \marginpar{...} if you do
%% not need to align the top of the contents to any
%% \cvsection title in the "main" bar.
\cvsection[sidebar]{Experience}

\cvexp{Co-founder}{Arbejd.com}{ 2018 - present }
Job matching services for employers and job seekers in order to match candidates to relevant jobs using metaheuristics and mathematics. The product consists of a mobile app for job seekers, a website for employers and a backend to run the matching algorithms. Developed in Ionic, C\# and Angular.

\divider

\cvexp{Systems developer}{SEAS-NVE ApS}{ 2016 - present }
Development of workforce management mobile applications and electrical meter data management systems. Infrastructure based on Azure in an event-driven architecture. Programs are written in C\# and F\#.

\divider

\cvexp{Co-founder}{Metanance IVS}{ 2016 - 2019 }
Road maintenance budget optimization based on metaheuristics. The product was developed in Java, Clojure and Angular.

\divider

\cvexp{Systems developer}{KGH Customs Software A/S}{ 2013 - 2016 }
Development of the software system Butterfly. A software product for customs handling when trading goods. It is written in C\# in a client-server architecture based on REST. I was a developer and later on  architect and team lead.

\divider

\cvexp{Consultant}{Bouvet ASA}{ 2011 - 2013 }
Development of an accounting product named Descartes. It is written in C\# in a client-server architecture communicating by REST. It was launched in Norway, Iceland, Slovenia and Poland. I was part of the initial team and was involved in everything from project startup, architecture, infrastructure, quality assurance, source-control management, continuous integration and of course programming.

\divider

\cvexp{Self-employed}{2ndEffect (Norway)}{ 2009 - 2011 }
Various start-ups and development of homepages, small programs, server setup and consulting.

\divider

\cvexp{Systems developer}{Ramb\o ll Informatik A/S}{ 2008 - 2009 }
Development of medium-sized web-based systems and maintenance of big enterprise systems. Mainly in .NET and Java.

\divider

\cvexp{Self-employed systems developer}{2ndEffect (Denmark)}{ 2005 - 2008 }
Homepages, small programs and server setup. Setup of web-, mail- and database servers on various Linux distros. Programming in C++, PHP, .NET, SQL and client-side JavaScript, XHTML and CSS.

\clearpage

%% If the NEXT page doesn't start with a \cvsection but you'd
%% still like to add a sidebar, then use this command on THIS
%% page to add it. The optional argument lets you pull up the 
%% sidebar a bit so that it looks aligned with the top of the
%% main column.
% \addnextpagesidebar[-1ex]{page3sidebar}

\end{document}
